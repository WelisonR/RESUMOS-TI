\documentclass[12pt]{exam} % possibilidade de uso do cancelspace
\usepackage{libertine}
\usepackage[utf8]{inputenc}
\usepackage[margin=1in]{geometry}
\usepackage{amsmath,amssymb,amsthm}
\usepackage{multicol}
\usepackage[brazil]{babel}
\usepackage[shortlabels]{enumitem}
\usepackage{csquotes}
\usepackage{color}

% ---
% Informações que podem ser configuradas (MACROS)
% ---
\newcommand{\class}{Universidade de Brasilia, Faculdade do Gama} % Nome da disciplina
\newcommand{\term}{2018.2}              % Perído Letivo
\newcommand{\examnum}{Matemática Discreta 01}      % Número/Nome do exercício.
\newcommand{\examdate}{12/12/2018}        % insere a data no documento
\newcommand{\estudant}{Welison Lucas A. Regis}
\newcommand{\register}{XX/XXXXXXX}
% \newcommand{\timelimit}{}
% ---



\begin{document} % declaração de que o documento começa aqui.
\pagestyle{plain}
\thispagestyle{empty}
% ... formatação do cabeçalho
\noindent
\begin{tabular*}{\textwidth}{l @{\extracolsep{\fill}} r @{\extracolsep{6pt}} l}
\textbf{\class} & \textbf{Nome:} & \textit{\estudant}\\
\textbf{\term} & \textbf{Matricula:} & \textit{\register}\\
\textbf{\examnum} &&\\
\textbf{\examdate} &&\\
\end{tabular*}\\
\rule[2ex]{\textwidth}{2pt}
% ---

\renewcommand{\solutiontitle}{\noindent\textbf{Solução: }\par\noindent}
\SolutionEmphasis{\color{red}}


% \printanswers
% Inicio das questões. 
\begin{questions}



\question Raciocinando semânticamente, determine a validade ou invalidade nos casos a seguir.
 
\begin{multicols}{2}
  \begin{enumerate}[(a)]

    \item 
     $A \lor B, \neg A \vDash B$

    \item
     $A \leftrightarrow B, \neg A \vDash \neg B$

    \item
     $\neg \left( A\land B \right)\vDash \neg B\land \neg A$

    \item
     $A\rightarrow B\vDash A\lor B$

    \item
     $\neg A\rightarrow \neg B\vDash A\rightarrow B$

    \item
     $A, A\rightarrow B\vDash A\leftrightarrow B$

    \item
     $B\rightarrow \neg C\vDash \neg(B\land C)$

    \item
     $\neg(A\lor B), C\leftrightarrow A\vDash \neg C$

  \end{enumerate}
\end{multicols}

\question Jogando-se dois dados, qual a probabilidade da soma ser 3?
\newline
% choices
% oneparchoices
\begin{oneparcheckboxes}
    \choice 3/36
    \CorrectChoice 2/36
    \choice 1/36
    \choice NDA
\end{oneparcheckboxes}

\question Descreva matemática as implicações lógicas Modus Ponens (MP) e Modus Tollens (MT).
\fillwithdottedlines{2cm}
%\fillwithlines{\stretch{1}}


\question Três moedas são lançadas ao mesmo tempo. Qual é a probabilidade de as três moedas caírem com a mesma face para cima? \answerline[Short answer]
\begin{solution}[2cm]
    ...
\end{solution}


\question Determine abaixo todas as permutações possíveis da palavra \enquote{BOBS}.
% \makeemptybox{1.5cm}

\begin{solutionorbox}[1.5cm]
    ...
\end{solutionorbox}

% solutionorlines
% solutionordottedlines
% solutionorgrid

\question Sobre análise combinatória, responda:
    \begin{parts}
        \part O que é um evento certo?

        \part
        \begin{subparts}
            \subpart O que é um espaço amostral? \answerline
            \subpart O que é um espaço amostral equidistante? \answerline
        \end{subparts}
    \end{parts}

\question \fillin[Probabilidade] é definido como a razão entre casos favoráveis e o espaço amostral. 

\newcommand{\tf}[1][{}]{
    \fillin[#1][0.25in]
}
\question Julge em verdadeiro ou falso:
    \begin{parts}
        \item \tf[T] A probabilidade de ocorrência de uma face qualquer de um dado não viciado é 1/6.
        \item \tf[F] A probabilidade de cair 5 ou 6 em um dado não viciado é 2/36.
    \end{parts}



\end{questions}

\vspace{1cm}
\noindent \textbf{RESPOSTAS}

Exemplo I.

\bigskip

a) $A \lor B, \neg A \vDash B$

\begin{proof}
Iremos demonstrar que o presente argumento é válido. Suponha, por absurdo, que o argumento é inválido. Assim, há uma valoração $v$, tal que:
i. $v(A\lor B)=V$, 
ii. $v(\neg A)=V$ e 
iii. $v(B)=F$. Note que de i. e iii., pelo significado da ($\lor$), temos que iv. $v(A)=V$. De iv., pelo significado da ($\neg$), temos que v. $v(\neg A)=F$. Contudo, de ii. e v., obtemos uma contradição, visto que $v$ é função. Segue-se disso que não há valoração que torne as premissas verdadeiras e a conclusão falsa. Portanto, o argumento é válido.\\
\end{proof}

\end{document}
